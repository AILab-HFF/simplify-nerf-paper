% !TEX root = ../main.tex
%
\chapter{Introduction}
\label{sec:intro}

View synthesis has long promised to revolutionize visual content creation in the film industry, but it wasn't until recent advancements in machine learning that this technology started to fulfill its potential.
The advent of Neural Radiance Fields (NeRF) has been pivotal, offering unprecedented realism and detail in 3D scene modeling and rendering.
This has significantly impacted various applications, from enhancing virtual reality experiences to the creation of photorealistic visual effects in movies.

Two prominent NeRF frameworks with user interfaces, namely Instant NGP \cite{muller_instant_2022} and Nerfstudio \cite{tancik_nerfstudio_2023}, have emerged as leaders in enabling users to explore and manipulate 3D scenes. 
These frameworks offer features such as real-time scene rendering, adjustable training parameters, and the creation of camera trajectories for video rendering.

Despite these advancements, the technical complexity of these frameworks frequently acts as a barrier to broader accessibility, indicating a need for improvements in user experience. 
These interfaces usually require a high degree of technical knowledge, as they are intended to supplement, rather than substitute, command-line interfaces.
For activities like video data preprocessing, model training, and output rendering, users are required to navigate through terminal-based processes.

This complexity not only limits the potential user base to individuals with technical expertise but also hinders the creative and innovative application of NeRF technology across broader fields. 
Consequently, there exists a critical need to enhance the user experience and develop solutions that simplify the interaction with NeRF frameworks, making them more approachable and usable for a diverse range of users beyond the realm of technical specialists.


\section*{Research Objectives and Questions}
\label{sec:intro:objectives_questions}

This research aims to advance the capabilities and accessibility of NeRF frameworks, making them accessible to a broader audience and fostering innovation in 3D scene modeling and rendering. The study is guided by the following questions:

\begin{enumerate}
    \item \textbf{Capabilities of Current NeRF Frameworks:} What are the existing interaction capabilities of NeRF frameworks, and how do they support user engagement, visualization, and manipulation of 3D scenes?

    \item \textbf{Enhancing Accessibility with a Web-Based Editor:} How can the development of a user-friendly web-based interface for NeRF improve its accessibility and simplify the creation and manipulation processes?

    \item \textbf{Challenges in Developing Web-Based NeRF Tools:} What are the primary technical challenges and limitations associated with the current NeRF frameworks and interfaces, and how can these be overcome?
\end{enumerate}

\section*{Scope of the Study}
\label{sec:intro:scope}
This research focuses on the development and evaluation of a web-based interface for NeRF, aiming to improve its accessibility and usability. 
The study will concentrate on interface design, user interaction, and the integration of editing functionalities, without delving into the underlying algorithms of NeRF technology itself. 
It is delimited by its emphasis on interface design over algorithmic advancements in NeRF processing.

\section*{Significance of the Study}
\label{sec:intro:significance}
By addressing the usability challenges of current NeRF frameworks, this research aims to make 3D scene modeling more accessible, fostering innovation and broadening the application of this technology across various fields. 
The development of a web-based interface could significantly lower the entry barrier to NeRF, enabling artists, designers, and educators to leverage this technology without requiring deep technical expertise.

\todo{add structure of the thesis}
% \section*{Structure of the Thesis}
% \label{sec:intro:structure}
% The thesis is structured as follows:

% \textbf{Chapter 2: Related Works} reviews existing NeRF frameworks, user interfaces, and editing tools, providing context for this research.
% \textbf{Chapter 3: Methodology} details the research methods employed in developing and evaluating the web-based interface.
% Subsequent chapters will cover the development process, evaluation results, discussion, and conclusions, culminating in recommendations for future research.