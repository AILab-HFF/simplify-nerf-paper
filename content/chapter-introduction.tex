% !TEX root = ../main.tex
%
\chapter{Introduction}
\label{sec:intro}

% \section{Background}
% \label{sec:intro:background}

Neural Radiance Fields (NeRF) have emerged as a transformative technology in 3D scene modeling and rendering, offering unprecedented realism and detail. 
This advancement has significantly impacted various applications, from virtual reality to cultural heritage preservation. 

Two prominent NeRF frameworks with user interfaces, namely Instant NGP \cite{muller_instant_2022} and Nerfstudio \cite{tancik_nerfstudio_2023}, have emerged as leaders in enabling users to explore and manipulate 3D scenes. 
These frameworks offer features such as real-time scene rendering, adjustable training parameters, and the creation of camera trajectories for video rendering.

Additionally, several innovative projects have expanded the NeRF landscape. Notably, CLIP-NeRF \cite{wang_clip-nerf_2022}, Instruct-NeRF2NeRF \cite{haque_instruct-nerf2nerf_2023}, Text2LIVE \cite{bar-tal_text2live_2022}, and SINE \cite{bao_sine_2023} have introduced text-based editing approaches, broadening the possibilities for manipulating NeRF models. PaletteNeRF \cite{wu_palettenerf_2022} focuses on color editing, while NeRF-Editing \cite{yuan_nerf-editing_2022} enables mesh editing. 

Despite these advancements, the technical complexity of these frameworks frequently acts as a barrier to broader accessibility, indicating a need for improvements in user experience. 
These interfaces usually require a high degree of technical knowledge, as they are intended to supplement, rather than substitute, command-line interfaces.
For activities like video data preprocessing, model training, and output rendering, users are required to navigate through terminal-based processes.

This complexity not only limits the potential user base to individuals with technical expertise but also hinders the creative and innovative application of NeRF technology across broader fields. 
Consequently, there exists a critical need to enhance the user experience and develop solutions that simplify the interaction with NeRF frameworks, making them more approachable and usable for a diverse range of users beyond the realm of technical specialists.


\section*{Research Objectives}
\label{sec:intro:objectives}

The research objectives of this study are as follows:

\begin{enumerate}
    \item \textbf{Exploration of NeRF Interaction Capabilities}: This study aims to explore the existing interaction capabilities within NeRF frameworks comprehensively. It involves an analysis of the current state of NeRF interfaces and an investigation into user engagement, visualizations, and manipulation of NeRF scenes.

    \item \textbf{Development of a Web-Based User Interface}: Building on insights gained from the exploration phase, the primary objective is to design and implement a user-friendly web-based interface for NeRF.

    \item \textbf{Streamlined NeRF Creation and Manipulation}: The central goal is to simplify the process of NeRF creation and manipulation, eliminating the need for users to deal with complex command-line interfaces or extensive local setup. The web-based interface will provide an intuitive and efficient user experience.

    \item \textbf{Integration of Diverse Editing Plugins}: To enhance the creative potential of NeRF, various editing plugins will be integrated into the web-based interface. The objective is to expand the functionality and versatility of the NeRF framework.
\end{enumerate}

The research aims to advance NeRF frameworks' capabilities and accessibility, making them accessible to a broader audience and fostering innovation in 3D scene modeling and rendering.


\section*{Research Question}
\label{sec:intro:question}

This research is guided by the following questions:

\begin{enumerate}
    \item \textbf{Enhancing NeRF Frameworks:} How can a web-based interface improve the user experience and accessibility of NeRF frameworks, and what impact will these enhancements have on user-friendly NeRF creation and manipulation?

    \item \textbf{Overcoming Technical Challenges:} What technical challenges and limitations are associated with current NeRF frameworks and interfaces, and how can innovative design and technology choices in a web-based interface overcome these challenges?

    \item \textbf{Innovative Editing Integration:} How can novel editing approaches be seamlessly integrated into a web-based NeRF interface to enhance creativity and usability, and how do these methods compare with traditional NeRF editing techniques?
\end{enumerate}

\section*{Scope of the Study}
\label{sec:intro:scope}
This research focuses on the development and evaluation of a web-based interface for NeRF, aiming to improve its accessibility and usability. 
The study will concentrate on interface design, user interaction, and the integration of editing functionalities, without delving into the underlying algorithms of NeRF technology itself. 
It is delimited by its emphasis on interface design over algorithmic advancements in NeRF processing.

\section*{Significance of the Study}
\label{sec:intro:significance}
By addressing the usability challenges of current NeRF frameworks, this research aims to make 3D scene modeling more accessible, fostering innovation and broadening the application of this technology across various fields. 
The development of a web-based interface could significantly lower the entry barrier to NeRF, enabling artists, designers, and educators to leverage this technology without requiring deep technical expertise.

% TODO: add structure of the thesis
% \section*{Structure of the Thesis}
% \label{sec:intro:structure}
% The thesis is structured as follows:

% \textbf{Chapter 2: Related Works} reviews existing NeRF frameworks, user interfaces, and editing tools, providing context for this research.
% \textbf{Chapter 3: Methodology} details the research methods employed in developing and evaluating the web-based interface.
% Subsequent chapters will cover the development process, evaluation results, discussion, and conclusions, culminating in recommendations for future research.