% !TEX root = ../main.tex
%
\chapter{Introduction}
\label{sec:intro}

\section{Motivation}
\label{sec:intro:motivation}

The potential of view synthesis to revolutionize visual content creation in the film industry has been recognized for some time. Nevertheless, it was not until recent advancements in machine learning that this technology began to fulfill its promise.
The advent of Neural Radiance Fields (NeRF) has been pivotal, offering unprecedented realism and detail in 3D scene modeling and rendering.
This has had a significant impact on various applications, including the enhancement of virtual reality experiences and the creation of photorealistic visual effects in movies.

Two prominent NeRF frameworks with user interfaces, namely Instant NGP \cite{muller_instant_2022} and Nerfstudio \cite{tancik_nerfstudio_2023}, have emerged as leaders in enabling users to explore and manipulate 3D scenes. 
These frameworks provide a number of features, including real-time scene rendering, adjustable training parameters, and the creation of camera trajectories for video rendering.

Despite these advancements, the technical complexity of these frameworks frequently acts as a barrier to broader accessibility, indicating a need for improvements in user experience. 
These interfaces usually require a high degree of technical knowledge, as they are intended to supplement, rather than substitute, command-line interfaces.
For activities such as video data preprocessing, model training, and output rendering, users are required to navigate through terminal-based processes.

This complexity not only limits the potential user base to individuals with technical expertise but also hinders the creative and innovative application of NeRF technology across broader fields. 
Consequently, there exists a critical need to enhance the user experience and develop solutions that simplify the interaction with NeRF frameworks, making them more accessible and usable for a diverse range of users beyond the realm of technical specialists.


\section{Research Objectives and Questions}
\label{sec:intro:objectives_questions}

This research aims to advance the capabilities and usability of NeRF frameworks, making them accessible to a broader audience and fostering innovation in 3D scene modeling and rendering. The study is guided by the following questions:

\begin{enumerate}
    \item \textbf{Capabilities of Current NeRF Frameworks:} What are the existing interaction capabilities of NeRF frameworks, and how do they support various user groups in creating and manipulating 3D scenes?

    \item \textbf{Enhancing Accessibility with a Web-Based Editor:} How can the development of a user-friendly web-based interface for NeRF improve its accessibility and simplify the creation and manipulation processes?

    \item \textbf{Challenges in Developing Web-Based NeRF Tools:} What are the primary technical challenges and limitations associated with building a NeRF interface and how can these be overcome?
\end{enumerate}

\paragraph{Scope of the Study}
This research project is focused on the development and evaluation of a web-based interface for NeRF, with the objective of enhancing its accessibility and usability.
The study will concentrate on interface design and user interaction, without delving into the underlying algorithms of NeRF technology itself.
It is delimited by its emphasis on interface design over algorithmic advancements in NeRF processing.

\paragraph{Significance of the Study}
This research addresses the usability challenges of current NeRF frameworks with the aim of making 3D scene modeling more accessible, fostering innovation and broadening the application of this technology across various fields.
The development of a web-based interface could significantly lower the entry barrier to NeRF, enabling artists, designers, and educators to leverage this technology without requiring deep technical expertise.

\section{Structure of the Thesis}
\label{sec:intro:structure}

This thesis is organized into several chapters, each focusing on a specific aspect of the research.
The structure is outlined as follows:

\begin{enumerate}
    \setcounter{enumi}{1}
    \item \textbf{Background:} Provides a comprehensive overview of view synthesis, including traditional and neural approaches, and specifically introduces Neural Radiance Fields for view synthesis.
    \item \textbf{Related Work:} Reviews existing solutions and technologies in the field, such as Instant NGP, NeRFStudio, Luma AI, and the Volinga Suite, discussing their features and limitations.
    \item \textbf{Methodology:} Describes the methodological approach used to research and develop the web-based interface for NeRF.
    \item \textbf{User Research:} Details the process and findings from user research, which informs the design and functionality of the developed interface.
    \item \textbf{Application Design:} Covers the design process of the application, from initial sketches to the final design, emphasizing user interface and experience.
    \item \textbf{Technical Implementation:} Explains the technical aspects of implementing the application, including system architecture and the integration of front-end and back-end components.
    \item \textbf{User Study and Evaluation:} Discusses the setup, execution, and results of user studies conducted to evaluate the usability and effectiveness of the interface.
    \item \textbf{Results:} Presents the findings from the user studies, analyzing the data collected through both quantitative and qualitative methods.
    \item \textbf{Discussion:} Reflects on the results, discussing implications for the film and VFX industry, and addressing the limitations encountered.
    \item \textbf{Conclusion:} Summarizes the key findings, contributions to the field, and suggests directions for future work.
\end{enumerate}
