\chapter{Conclusion}
\label{sec:conclusion}

\section{Key Findings}
\label{sec:conclusion:findings}

Neural Radiance Fields have emerged as a powerful technology for view synthesis, particularly for its ability to render photorealistic scenes with significant applicability in the film industry.
A review of existing NeRF interfaces has identified significant usability issues that are impeding the adoption of this technology by filmmakers.
Tools that are accessible to non-experts often lack the necessary control features, while those designed for experts present a steep learning curve due to their complexity.

To address these issues, this research developed a new NeRF interface that strikes a balance between simplicity and functional depth.
The graphical user interface streamlines the NeRF model creation process by consolidating essential controls.
The user study conducted to evaluate this interface demonstrated that it was well received by both novice and expert users.
The simplicity and guided process were appreciated by novices, while the extensive control options and workflow enhancements were appreciated by experts.
Nevertheless, the study also identified areas for potential enhancement, suggesting refinements to the interface's usability and the incorporation of additional features.

\section{Contributions to the Field}
\label{sec:conclusion:contributions}

The thesis makes a significant contribution to the field by conducting a general evaluation of NeRF tools. This highlights the practical limitations and needs associated with current NeRF technologies.
A novel interface has been developed, designed to effectively balance ease of use for newcomers with the depth required by advanced users.
Furthermore, the interface has been subjected to a rigorous evaluation through a user study, which has confirmed its enhanced usability and functionality.

\section{Future Work}
\label{sec:conclusion:future}

Although the new NeRF interface has demonstrated promising results, there is considerable scope for further development.
The interface could be enhanced to support a wider range of Nerfstudio features and plugins, thereby broadening the tool's capabilities.
Furthermore, it is necessary to enhance the system's robustness and scalability in order to accommodate an expanding user base and increasingly complex application scenarios.
It is imperative that the framework be more deeply integrated with Nerfstudio if its full capabilities are to be fully leveraged, thereby enhancing both usability and functionality \tref{sec:system:future}.
Finally, the continuous refinement of the interface, guided by ongoing user feedback and extended user testing, including comparative studies, will provide deeper insights and encourage continual improvements.
These future directions are of pivotal importance for maintaining the interface's relevance and effectiveness in adapting to the evolving demands of filmmakers and other creative professionals.