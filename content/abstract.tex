% !TEX root = ../main.tex
%

\pdfbookmark[0]{Abstract}{Abstract}
{\usekomafont{chapter}Abstract}\label{sec:abstract} \\
\vspace*{-2mm}

The advent of Neural Radiance Fields (NeRF) has transformed the field of 3D scene modelling and rendering, offering photorealistic digital environments at relatively low effort and cost.
Despite the technology's potential, the complexity of existing NeRF frameworks presents significant challenges to users without extensive technical expertise.
This thesis presents the development of a web-based interface designed to simplify the creation of NeRFs, making them accessible to a broader audience, specifically targeting professionals from the film industry.
The interface has been designed to facilitate the use of NeRFs by those without prior technical expertise, while still allowing users to leverage the full capabilities of the technology.
The research employs a user-centred design approach, based on expert interviews to gauge user requirements and preferences.
A subsequent user study evaluated the interface through qualitative and quantitative methods, in order to validate its effectiveness and accessibility.
The contributions of this work not only enhance the user experience but also expand the creative possibilities for users across various use cases, thereby promoting broader adoption and innovation in NeRF-based content creation.

\vspace*{10mm}

{\usekomafont{chapter}Abstract (German)}\label{sec:abstract-diff} \\
\vspace*{-2mm}

Neural Radiance Fields (NeRF) haben die Bereiche der 3D-Modellierung ma{\ss}geblich verändert.
Sie ermöglichen die Erstellung fotorealistischer digitaler Szenen bei relativ geringem Aufwand und Kosten.
Trotz des Potenzials der Technologie stellen die Komplexität der bestehenden NeRF-Frameworks sowie die erforderlichen technischen Kenntnisse für deren Nutzung eine Herausforderung für Benutzer ohne entsprechende Vorkenntnisse dar.
In dieser Arbeit wird die Entwicklung einer webbasierten Oberfläche vorgestellt, die die Erstellung von NeRFs vereinfachen soll, um sie einem breiteren Publikum zugänglich zu machen, insbesondere Fachleuten aus der Filmindustrie.
Die Schnittstelle wurde so konzipiert, dass sie auch Personen ohne technische Vorkenntnisse die Verwendung von NeRFs erleichtert und es den Nutzenden dennoch ermöglicht, die Kapazitäten der Technologie voll auszuschöpfen.
Die Forschungsarbeit folgt einen nutzerzentrierten Designansatz, basierend auf Experten Interviews um die Anforderungen und Präferenzen der Nutzer zu ermitteln.
In einer anschlie{\ss}enden Nutzerstudie wurde die Oberfläche mit Hilfe qualitativer und quantitativer Methoden bewertet, um ihre Wirksamkeit und Zugänglichkeit zu validieren.
Die Beiträge dieser Arbeit verbessern nicht nur das Benutzererlebnis, sondern erweitern auch die kreativen Möglichkeiten für Benutzer in verschiedenen Anwendungsfällen und fördern so eine Verbreitung und Innovation bei der Erstellung von NeRF-basierten Inhalten.