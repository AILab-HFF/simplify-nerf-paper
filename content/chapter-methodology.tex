% !TEX root = ../main.tex
%
\chapter{Methodology}
\label{sec:methodology}

The research was structured into three distinct phases: initial user research, prototype development, and user testing.
Each phase was designed to inform and refine the subsequent stages, ensuring a systematic approach to developing a user-friendly NeRF interface.
The iterative process was designed to align closely with user needs and feedback, with the objective of creating a design that is both accessible and functional.

\cleanparagraph{Initial User Research}
The initial phase of this research involved conducting a series of in-depth interviews with users of NeRF technology.
The objective of these interviews was to gain insight into the user experience of NeRF technology.
The interviews were conducted with users of varying levels of expertise in NeRF model creation, including those with experience in the film industry.
This exploratory phase was crucial for identifying the key features and improvements necessary for a more accessible and efficient NeRF interface.
(see Chapter~\ref{sec:user-research})

\cleanparagraph{Design and Development Process}
The transition from the initial findings of user research to the creation of a functional prototype was a multi-step process focused on the capture of user needs and their subsequent translation into a tangible design.
This phase involved the creation of a user flow diagram, site map, wireframes, and a working prototype, each of which was informed by the insights gained from the previous stage.
(see Section~\ref{sec:design:ux})

\cleanparagraph{User Study and Evaluation}
To assess the usability and overall utility of the developed prototype, a comprehensive user study was conducted.
The primary objective of this study was to gather feedback on the prototype's user experience, identify any challenges participants encountered, and assess their levels of satisfaction with the interface.
The use of a mixed-methods approach permitted the collection and analysis of both quantitative and qualitative data, providing a multifaceted view of the prototype's performance in real-world tasks.
(see Chapter~\ref{sec:study})
