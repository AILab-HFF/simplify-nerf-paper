% !TEX root = ../main.tex
%
\chapter{Methodology}
\label{sec:methodology}

This research was organized into three sequential phases: initial user research, prototype development, and user testing. 
Each phase was designed to inform and refine the subsequent stages, ensuring a systematic approach to developing a user-friendly NeRF interface. 
This iterative process aimed to align closely with user needs and feedback, creating a design that is both intuitive and functional.

\paragraph{Initial User Research}

The foundational stage of this research involved conducting a series of in-depth interviews to gather insights into the user experience of NeRF technology. 
The primary objective was to understand the varied challenges, needs, and preferences of users, ranging from novices to experts in NeRF model creation, particularly those with ties to the film industry. 
This exploratory phase was crucial for identifying the key features and improvements necessary for a more accessible and efficient NeRF interface. (see Chapter~\ref{sec:user-research})

\paragraph{Design and Development Process}

The transition from initial user research findings to a functional prototype was a multi-step process focused on capturing user needs and translating them into a tangible design. 
This phase involved the creation of a user flow diagram, site map, wireframes, and a working prototype, each building upon the insights gained from the previous stage. (see Section~\ref{sec:design:ux})

\paragraph{User Study and Evaluation}

To evaluate the usability and overall utility of the developed prototype, a comprehensive user study was conducted. 
The primary aim of this study was to collect feedback on the prototype's user experience, identify any challenges participants encountered, and gauge their levels of satisfaction with the interface. 
Employing a mixed-methods approach allowed for a blend of quantitative and qualitative data collection and analysis, offering a multifaceted view of the prototype's performance in real-world tasks. (see Chapter~\ref{sec:study})
