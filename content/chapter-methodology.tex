% !TEX root = ../main.tex
%
\chapter{Methodology}
\label{sec:methodology}

This research was organized into three sequential phases: initial user research, prototype development, and user testing. 
Each phase was designed to inform and refine the subsequent stages, ensuring a systematic approach to developing a user-friendly NeRF interface. 
This iterative process aimed to align closely with user needs and feedback, fostering a design that is both intuitive and functional.

\section{Initial User Research}
\label{sec:methodology:user-research}

The foundational stage of this research involved conducting a series of in-depth interviews to gather insights into the user experience of NeRF technology. 
The primary objective was to understand the varied challenges, needs, and preferences of users, ranging from novices to experts in NeRF model creation, particularly those with ties to the film industry. 
This exploratory phase was crucial for identifying the key features and improvements necessary for a more accessible and efficient NeRF interface.

\subsection*{Participant Selection Criteria}
\label{sec:methodology:user-research:criteria}

Participants were carefully chosen based on their prior experience with NeRF technology and their connection to the film industry, culminating in a group of four experts. 
This selection ensured a rich diversity of perspectives, encompassing a broad spectrum of technical proficiency and practical applications of NeRF. 
By including individuals who have utilized NeRF in various capacities, the study aimed to uncover both the shared challenges faced by all users and the unique requirements of distinct user groups within the film industry.

\subsection*{Interview Methodology}
\label{sec:methodology:user-research:interview}

The interviews were designed as semi-structured conversations, following a core set of predetermined questions (see Appendix \ref{sec:appendix:interview-questions}) while also allowing for spontaneous discussions and additional queries. 
Conducted one-on-one, these interviews facilitated a personalized dialogue with each participant, offering insights into individual experiences and perspectives. 
Although the interviews were prepared in English, all conversation were held in German, to ensure comfort and clarity for participants, potentially leading to more candid and informative discussions.

The structured flow of questions began with learning about the participants' backgrounds and experiences with NeRF technology, gradually moving towards more detailed questions about their specific needs, challenges, and desired improvements in NeRF interfaces. 
Participants were also invited to propose features or functionalities they believed would enhance the usability and effectiveness of a NeRF interface for their professional or academic projects.

To ensure a comprehensive analysis, interviews were recorded and transcribed with participants' consent, allowing for a detailed review and coding of the responses. 
This process enabled the identification of recurring themes, challenges, and preferences across the participant group, providing a solid foundation for the subsequent phases of prototype development and user testing. 
The insights gained from this initial research phase were instrumental in shaping the direction and focus of the interface design, ensuring it would effectively address the real-world needs of NeRF users.

\subsection*{Key Findings}
\label{sec:methodology:user-research:findings}

\paragraph{NeRF in the Film Industry}

NeRF technology is being explored for various applications in the film industry, such as visual effects, virtual production, and pre-production location scouting. 
Despite its potential to simplify the creation of 3D scenes, current limitations in model quality, lack of editable models, and insufficient detail hinder its professional use. 
However, its capability for quick 3D scene captures offers significant benefits for pre-visualization and planning in the pre-production phase, although concerns about model scale accuracy for export remain.

\paragraph{Optimizing Parameters and Workflow}

Creating NeRFs typically involves preprocessing input data, training models, and exporting outputs. 
Technical users emphasize the importance of parameter optimization in improving NeRF quality, with iterative training and results analysis being crucial parts of their workflow. 
Tools like TensorBoard are utilized for quantifying variations in training outcomes.

\paragraph{User Interface and Accessibility}

A consensus among users highlights the need for an intuitive, all-encompassing user interface that minimizes reliance on console commands. 
Features that allow users to visually navigate and control the NeRF creation pipeline, including real-time progress feedback and the ability to pause and adjust processes at any stage, are highly valued.

\paragraph{Web-Based Interfaces and Collaboration}

Preferences have shifted towards web-based interfaces, facilitating remote project management and data handling. 
Such interfaces support collaborative efforts, allowing users to easily share and review project stages.

\paragraph{Comprehensive Error Handling and Visualization}

Effective error feedback and clear, informative visualization tools are critical for user satisfaction. 
Users express frustration with vague error messages and cumbersome command-line interactions for troubleshooting and adjustments.

\paragraph{File Management and Project Structure}

Efficient file and project management, with clear distinctions between different stages (preprocessing, training, rendering) and support for various input formats, is essential. 
Users discuss challenges with current tools regarding data organization, suggesting improvements for handling input data and managing projects​​.

% \paragraph{Quality, Performance, and Scalability}

% Concerns about the scalability of projects, particularly when dealing with large datasets or aiming for high-quality outputs, are prevalent. 
% Users desire improvements in performance management and quality assurance mechanisms.

\paragraph{Integration and Export Options}

Strong integration capabilities with popular 3D and VFX software and flexible export options are desired. 
Users discuss the importance of being able to easily import NeRF-generated scenes into tools like Unreal Engine or Blender for further processing and use in production-quality projects​​.

\paragraph{Multi-Mode Operation}

The necessity for multi-mode operation in NeRF tool interfaces emerges as a significant insight, underlining the importance of accommodating a broad spectrum of users, from novices to experts. 
A simplified mode is envisioned to cater to beginners, offering an intuitive and streamlined workflow, whereas an advanced mode is tailored for experienced users requiring detailed control over the NeRF creation process. 
% By embracing a multi-mode approach, NeRF tools can democratize access to sophisticated 3D modeling capabilities, fostering a creative and inclusive environment for users of all backgrounds.

\paragraph{Conclusion}

These findings highlight the demand for a NeRF tool interface that is user-friendly, versatile, and capable of supporting a wide range of workflows and user expertise levels. 
The ideal tool would combine intuitive project management and visualization features with powerful customization options, robust error handling and feedback mechanisms, and effective performance management capabilities.


\section{User Study and Evaluation}
\label{sec:methodology:study}

To evaluated the usability and usefulness of the developed prototype, a user study was conducted. 
The study was designed to gather feedback on the user experience of the prototype, and to identify any usability issues or challenges that users encountered.
It followed a mixed-methods approach, combining quantitative and qualitative data collection and analysis. 
Participants were given a set of tasks to complete with the prototype, after which they fill out a questionnaire and participate in a short follow-up interview.

\subsection*{Tasks Based Usability Test}
\label{sec:methodology:study:tasks}

The usability test was conducted in a controlled environment, with participants being asked to complete a two tasks with the prototype.
The tasks were designed to cover a range of functionalities and features of the prototype, and represent a typical workflow when creating NeRF models.

\begin{enumerate}
  \item Task: Create a new project.
  \item Task: Upload a prepared video file.
  \item Task: Pre-process the uploaded file to prepare it for training.
  \item Task: Switch to an existing project train a NeRF that already pre-processed data.
  \item Task: Start a NeRF training.
  \item Task: Create a camera path in the viewer.
  \item Task: Export a video.
\end{enumerate}

To keep an appropriate time frame, none of the tasks required completion of a training process, and pre-processed data and pre-trained models were provided to the participants.
On average, participants took 30 minutes to complete the tasks. % TODO: check time

Participant were passively observed while working on their tasks, to identify any problems or operation errors they encountered and to determine their overall performance.
In addition the screen was recorded to capture the participants interactions with the prototype, and to allow for a more detailed analysis of their behavior later on.

\subsection*{User Experience Questionnaire}
\label{sec:methodology:study:ueq}

After completing their tasks, users were asked to fill out the User Experience Questionnaire (UEQ) \cite{laugwitz_construction_2008}, a standardized questionnaire for the assessment of user experience.
It measure user experience in six dimensions:

\begin{itemize}
  \item \textbf{Attractiveness} - the overall impression of the product
  \item \textbf{Perspicuity} - the clarity and understandability of the product
  \item \textbf{Efficiency} - the perceived effort required to use the product
  \item \textbf{Dependability} - the perceived reliability and trustworthiness of the product
  \item \textbf{Novelty} - the perceived originality and innovation of the product
  \item \textbf{Stimulation} - the perceived level of excitement and engagement with the product
\end{itemize}

This covers both classical usability goals (Efficiency, Perspicuity, Dependability) and user experience qualities (Novelty, Stimulation).
Attractiveness is purely a valence dimension, and is not directly related to usability or user experience.

In total the questionnaire consists of 26 items, each represented by two terms of opposite meaning. 
The order of the terms is randomized for each item, to avoid bias.
Participants are asked to rate each item on a 7-point scale, from -3 to +3, with 0 representing a neutral response.
An example of the scale can be seen below:

\begin{center}
  boring \quad o o o o o o o \quad exciting
\end{center}

The format of the questionnaire gives participants a clear and simple way to  quickly express their feelings and thoughts about the prototype, without much effort.

In this study the questionnaire was filled out by participants in digital form, using a web-based survey tool. %TODO: source
The survey included additional questions to gather demographic information and to capture prior experience with NeRF and other 3D modeling tools.
This allowed for a more efficient data collection and analysis, across in-person and remote participants.

\subsection*{Follow-up Interview}
\label{sec:methodology:study:interview}

After completing the usability test, participants were engaged in a short follow-up interview, to gather more detailed feedback on their experience with the prototype. 
Similar to the initial user interviews, these interviews were semi-structured, following a pre-defined set of question, with room for participants to share their own thoughts and suggestions.
The questions can be categorized into general usability, tasks specific feedback and suggestions for improvement.
The interview template can be found in the appendix \ref{sec:appendix:questionnaire}.


\subsection*{Data Analysis}
\label{sec:methodology:study:analysis}

Both the recordings of the usability test and the follow-up interviews were analyzed to identify common themes and patterns in the feedback of participants.
The video recordings were coded to highlight any usability issues or challenges that participants encountered during the tasks.
The audio recordings of the interviews were transcribed and coded.
The data was then categorized and analyzed to identify common themes and patterns across the participants.

Analysis of the UEQ data was done using the standard procedure for the questionnaire.
The UEQ provides a data analysis tool in form of spreadsheet, that calculates all necessary values and visualizes the results.
